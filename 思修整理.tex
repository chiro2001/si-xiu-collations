% Options for packages loaded elsewhere
\PassOptionsToPackage{unicode}{hyperref}
\PassOptionsToPackage{hyphens}{url}
%
\documentclass[
]{article}
\usepackage{lmodern}
\usepackage{amssymb,amsmath}
\usepackage{ifxetex,ifluatex}
\ifnum 0\ifxetex 1\fi\ifluatex 1\fi=0 % if pdftex
  \usepackage[T1]{fontenc}
  \usepackage[utf8]{inputenc}
  \usepackage{textcomp} % provide euro and other symbols
\else % if luatex or xetex
  \usepackage{unicode-math}
  \defaultfontfeatures{Scale=MatchLowercase}
  \defaultfontfeatures[\rmfamily]{Ligatures=TeX,Scale=1}
\fi
% Use upquote if available, for straight quotes in verbatim environments
\IfFileExists{upquote.sty}{\usepackage{upquote}}{}
\IfFileExists{microtype.sty}{% use microtype if available
  \usepackage[]{microtype}
  \UseMicrotypeSet[protrusion]{basicmath} % disable protrusion for tt fonts
}{}
\makeatletter
\@ifundefined{KOMAClassName}{% if non-KOMA class
  \IfFileExists{parskip.sty}{%
    \usepackage{parskip}
  }{% else
    \setlength{\parindent}{0pt}
    \setlength{\parskip}{6pt plus 2pt minus 1pt}}
}{% if KOMA class
  \KOMAoptions{parskip=half}}
\makeatother
\usepackage{xcolor}
\IfFileExists{xurl.sty}{\usepackage{xurl}}{} % add URL line breaks if available
\IfFileExists{bookmark.sty}{\usepackage{bookmark}}{\usepackage{hyperref}}
\hypersetup{
  hidelinks,
  pdfcreator={LaTeX via pandoc}}
\urlstyle{same} % disable monospaced font for URLs
\setlength{\emergencystretch}{3em} % prevent overfull lines
\providecommand{\tightlist}{%
  \setlength{\itemsep}{0pt}\setlength{\parskip}{0pt}}
\setcounter{secnumdepth}{-\maxdimen} % remove section numbering
\ifluatex
  \usepackage{selnolig}  % disable illegal ligatures
\fi

\author{}
\date{}

\begin{document}

\hypertarget{header-n0}{%
\section{思修整理}\label{header-n0}}

\hypertarget{header-n2}{%
\subsection{绪论}\label{header-n2}}

\hypertarget{header-n3}{%
\subsubsection{大学生作为时代新人的要求是什么?}\label{header-n3}}

\begin{enumerate}
\def\labelenumi{\arabic{enumi}.}
\item
  有理想
\item
  有本领
\item
  有担当
\end{enumerate}

\hypertarget{header-n11}{%
\subsection{第一章:人生的青春之间}\label{header-n11}}

\hypertarget{header-n12}{%
\subsubsection{人生观(是什么)}\label{header-n12}}

\begin{itemize}
\item
  \textbf{什么是}人生观?人们关于人生目的
  、人生态度、人生价值等问题的总观点和总看法。
\item
  人的\textbf{本质}是什么?人的本质是一切社会关系的总和。
\item
  人生观的主要内容?

  \begin{enumerate}
  \def\labelenumi{\arabic{enumi}.}
  \item
    人生\textbf{目的}(最重要)
  \item
    人生\textbf{态度}
  \item
    人生\textbf{价值}

    \begin{enumerate}
    \def\labelenumii{\arabic{enumii}.}
    \item
      \textbf{自我}价值
    \item
      \textbf{社会}价值(更重要)
    \end{enumerate}
  \item
    \textbf{上面三者的关系:}相互联系,相辅相成,(辩证统一)。
  \end{enumerate}
\item
  世界观和人生观的关系?世界观\textbf{决定}人生观。
\item
  什么是正确的人生观?``服务人民、奉献社会''的思想,代表了人类社会迄今最先进的人生追求。
\item
  什么是积极进取的人生态度?\textbf{认真务实}、\textbf{乐观}向上、积极\textbf{进取}
\item
  如何评价人生价值?

  \begin{enumerate}
  \def\labelenumi{\arabic{enumi}.}
  \item
    根本尺度:一个人的实践活动是否符合社会发展的客观规律,是否促进了历史的进步。
  \item
    能力大小与贡献尽力相统一
  \item
    物质贡献与精神贡献统一
  \item
    完善自身与奉献社会统一
  \end{enumerate}
\end{itemize}

\hypertarget{header-n51}{%
\subsubsection{人生观(怎么做)}\label{header-n51}}

\begin{itemize}
\item
  如何对待处理人生矛盾?树立正确的【幸福观、得失观、苦乐观、顺逆观、生死观、荣辱观】。
\end{itemize}

\hypertarget{header-n55}{%
\subsection{第二章:坚定理想信念}\label{header-n55}}

\hypertarget{header-n56}{%
\subsubsection{理想信念}\label{header-n56}}

\begin{itemize}
\item
  理想信念的重要性:理想信念是人的精神世界的核心,是人精神上的``钙''。
\item
  为什么理想信念是精神的``钙''?因为理想信念能够

  \begin{enumerate}
  \def\labelenumi{\arabic{enumi}.}
  \item
    昭示奋斗目标
  \item
    提供前进动力
  \item
    提高精神境界
  \end{enumerate}
\end{itemize}

\hypertarget{header-n69}{%
\subsubsection{马克思主义}\label{header-n69}}

\begin{itemize}
\item
  大学生应该树立什么信仰?马克思主义的科学信仰。
\item
  为什么要信仰马克思主义?

  \begin{enumerate}
  \def\labelenumi{\arabic{enumi}.}
  \item
    马克思主义体现了科学性和革命性的统一
  \item
    马克思主义具有鲜明的实践品格
  \item
    马克思主义具有持久生命力
  \end{enumerate}
\end{itemize}

\hypertarget{header-n82}{%
\subsubsection{中国特色社会主义}\label{header-n82}}

\begin{itemize}
\item
  中国特色社会主义是我们的\textbf{共同理想}
\item
  中国特色社会主义\textbf{本质特征}:中国共产党的领导
\item
  正确认识共产主义远大理想和中国特色社会主义远大理想之间的关系
\end{itemize}

\hypertarget{header-n90}{%
\subsection{第三章:弘扬中国精神}\label{header-n90}}

\begin{itemize}
\item
  民族精神核心:爱国主义
\item
  时代精神核心:改革创新
\end{itemize}

\hypertarget{header-n96}{%
\subsubsection{爱国主义}\label{header-n96}}

\begin{itemize}
\item
  爱国主义基本内涵

  \begin{enumerate}
  \def\labelenumi{\arabic{enumi}.}
  \item
    爱祖国的大好河山
  \item
    爱自己的骨肉同胞
  \item
    爱祖国的灿烂文化
  \item
    爱自己的国家
  \end{enumerate}
\item
  新时代爱国主义

  \begin{enumerate}
  \def\labelenumi{\arabic{enumi}.}
  \item
    坚持爱国主义和社会主义相统一
  \item
    维护祖国统一和民族团结
  \item
    尊重和传承中华民族历史和文化
  \item
    必须立足民族又面向世界
  \end{enumerate}
\end{itemize}

\hypertarget{header-n120}{%
\subsubsection{改革创新}\label{header-n120}}

\begin{itemize}
\item
  改革创新是时代要求

  \begin{enumerate}
  \def\labelenumi{\arabic{enumi}.}
  \item
    创新始终是推动人类社会发展的第一动力
  \item
    创新能力是当今国际竞争新优势的集中体现
  \item
    改革创新是我国赢得未来的必然要求
  \end{enumerate}
\item
  如何做改革创新生力军?

  \begin{enumerate}
  \def\labelenumi{\arabic{enumi}.}
  \item
    树立改革创新的自觉意识
  \item
    增强改革创新的能力本领
  \end{enumerate}
\end{itemize}

\hypertarget{header-n154}{%
\subsection{第四章:践行社会主义核心价值观}\label{header-n154}}

\hypertarget{header-n156}{%
\subsubsection{坚定价值观自信}\label{header-n156}}

\begin{itemize}
\item
  怎样坚定价值观自信?

  \begin{enumerate}
  \def\labelenumi{\arabic{enumi}.}
  \item
    社会主义核心价值观的历史底蕴
  \item
    社会主义核心价值观的现实基础
  \item
    社会主义核心价值观的道义力量
  \end{enumerate}
\end{itemize}

\hypertarget{header-n179}{%
\subsubsection{践行社会主义核心价值观}\label{header-n179}}

\begin{itemize}
\item
  如何践行社会主义核心价值观?
\end{itemize}

\end{document}
